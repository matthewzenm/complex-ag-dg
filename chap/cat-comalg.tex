\chapter{Categorical Language}

In this chapter, we breifly intrduce the language of category theory.

\section{(Rapid) Definitions}

In this section we list some basic notions in category theory.
We shall use sans serif capital letter to denote categories.

\begin{defn}
    A \emph{category} $\mathsf{C}$ consists of a collection $\Obj\mathsf{C}$ of \emph{objects} and a set of \emph{morphisms} $\Hom_\mathsf{C}(A,B)$ for each $A,B\in\Obj{\mathsf{C}}$, together with a mapping called \emph{composition} for each $A,B,C\in\Obj\mathsf{C}$
    \[\circ:\Hom_\mathsf{C}(B,C)\times\Hom_\mathsf{C}(A,B)\to\Hom_\mathsf{C}(A,C)\]
    satisfying:
    \begin{enumerate}[(1)]
        \item For any $f\in\Hom_\mathsf{C}(A,B)$, $g\in\Hom_\mathsf{C}(B,C)$ and $h\in\Hom_\mathsf{C}(C,D)$, the \emph{associative law} $(h\circ g)\circ f=h\circ(g\circ f)$ holds;
        \item There exists an $\id_A\in\Hom_\mathsf{C}(A,A)$ for each $A\in\Obj\mathsf{C}$, satisfiying for any $f\in\Hom_\mathsf{C}(A,B)$ and $g\in\Hom_\mathsf{C}(C,A)$ the equalities $f\circ\id_A=f$ and $\id_A\circ g=g$ hold.
    \end{enumerate}
    We often write $f:A\to B$ for $f\in\Hom_\mathsf{C}(A,B)$.
    If $\Obj\mathsf{C}$ is a set, we call $\mathsf{C}$ a \emph{small category}.
\end{defn}

\begin{eg}
    We have the following common categories.
    \begin{enumerate}
        \item $\mathsf{Set}$, objects are sets, morphisms are mappings.
        \item $\mathsf{Grp}$, objects are groups, morphisms are group homomorphisms.
        \item $\mathsf{Ab}$, objects are abelian groups, morphisms are group homomorphisms.
        \item $\mathsf{CRing}$, objects are commutative rings with multiplication identity element, morphisms are ring homomorphisms.
        We shall assume rings are commutative and with multiplication element through out the note.
        \item $\mathsf{Mod}_A$, objects are modules over ring $A$, morphisms are module homomorphisms.
        \item $\mathsf{Top}$, objects are topological spaces, morphisms are maps, i.e.\ continuous mappings.
    \end{enumerate}
    We will encounter more categories in our tour.
\end{eg}

\begin{defn}
    A morphism $f\in\Hom_\mathsf{C}(A,B)$ is called an \emph{isomorphism} if there exists a functor $g:B\to A$ such that $g\circ f=\id_A$ and $f\circ g=\id_B$.

    A morphism $i\in\Hom_\mathsf{C}(A,B)$ is called a \emph{monomorphism} if for any $f,g\in\Hom_\mathsf{C}(B,C)$, $i\circ g=i\circ h$ implies $g=h$.
    A morphism $e\in\Hom_\mathsf{C}(A,B)$ is called an \emph{epimorphism} if for any $f',g'\in\Hom_\mathsf{C}(D,A)$, $f'\circ e=g'\circ e$ implies $f'=g'$.
\end{defn}

\begin{defn}
    A \emph{(covariant) functor} $F:\mathsf{C}\to\mathsf{D}$ from category $\mathsf{C}$ to $\mathsf{D}$ consists of an assignment $F(A)$ to each $A\in\Obj\mathsf{C}$, together with an assignment $F(f)\in\Hom_\mathsf{D}(F(A),F(B))$ to $f\in\Hom_\mathsf{C}(A,B)$ for each $A,B\in\Obj\mathsf{C}$, which preseves identity and is distributive to composition, i.e.\ $F(\id_A)=\id_{F(A)}$ for $A\in\Obj\mathsf{C}$ and $F(g)\circ F(f)=F(g\circ f)$ for $f\in\Hom_\mathsf{C}(A,B)$ and $g\in\Hom_\mathsf{C}(B,C)$.
\end{defn}

\begin{defn}
    The \emph{opposite category} of a category $\mathsf{C}$, denoting $\mathsf{C}^{\mathrm{op}}$, consists of $\Obj\mathsf{C}^{\mathrm{op}}=\Obj\mathsf{C}$ and $\Hom_{\mathsf{C}^{\mathrm{op}}}(A,B)=\Hom_\mathsf{C}(B,A)$.
    A \emph{contravariant functor} $F:\mathsf{C}\to\mathsf{D}$ is a covariant functor $\mathsf{C}^{\mathrm{op}}\to\mathsf{D}$.
\end{defn}

\begin{defn}
    Let $F,G:\mathsf{C}\to\mathsf{D}$ be two functors, a \emph{natural transformation} from $F$ to $G$ is a family of isomorphisms $\{\varphi_A\in\Hom_{\mathsf{D}}(F(A),G(A))\}_{A\in\Obj\mathsf{C}}$, which satisfies that for any $A,B\in\Obj\mathsf{C}$ and $f\in\Hom_\mathsf{C}(A,B)$, the diagram
    \[\begin{tikzcd}
        F(A) \ar[r, "\varphi_A"] \ar[d, "F(f)"] & G(A) \ar[d, "G(f)"] \\
        F(B) \ar[r, "\varphi_B"] & G(B)
    \end{tikzcd}\]
    is commutative.
\end{defn}

\begin{eg}
    Let $\mathsf{C}$ be a small category and $\mathsf{D}$ be an arbitrary category, the functors from $\mathsf{C}$ to $\mathsf{D}$ together with natural transformations consists a category, we denote it by $\Fun(\mathsf{C},\mathsf{D})$.
    We ask $\mathsf{C}$ to be small in order to make the natural transformations between two functors consist a set.
\end{eg}

\begin{defn}
    Let $\mathsf{C}$ and $\mathsf{D}$ be two categories, $\mathsf{C}$ and $\mathsf{D}$ are called \emph{equivalent} if there exists functors $F:\mathsf{C}\to\mathsf{D}$ and $G:\mathsf{D}\to\mathsf{C}$ and natural isomorphisms $\varphi,\psi$, satisfiying $G\circ F\xrightarrow{\varphi}\id_\mathsf{C}$ and $F\circ G\xrightarrow{\psi}\id_\mathsf{D}$.
    If $\varphi$ and $\psi$ are identities, then we call $\mathsf{C}$ and $\mathsf{D}$ are \emph{isomorphic}.
\end{defn}

\section{Limits and Colimits}

In this section we introduce limits and colimits of functors.
We shall use limits and colimits to define products, coproducts, pushout, pullback and so on.
We shall also introduce equalizers and coequalizers, and the notion of completeness and cocompleteness.
Finally we shall focus on filtered colimits, i.e.\ direct limits.
We shall give concrete construction of filtered colimits, and some examples.

\begin{nota}
    Let $\mathsf{I}$ be a small category and $\mathsf{C}$ be an arbitrary category.
    We denote $\underline{A}\in\Fun(\mathsf{I},\mathsf{C})$ to be the functor that assigns every $i\in\Obj\mathsf{I}$ to $A$ and every morphism to $\id_A$.
\end{nota}

\begin{defn}
    Let $\mathsf{I}$ be a small category, $\mathsf{C}$ be an arbitrary category, and $F\in\Fun(\mathsf{I},\mathsf{C})$.
    The \emph{colimit} of $F$ is a functor $\underline{\colim{F}}$ for some $\colim{F}\in\Obj\mathsf{C}$ together with a natural transformation $\iota:F\to\underline{\colim{F}}$, charachterized by following universal property:
    For any natural transformation $F\to\underline{A}$ for some $A\in\Obj\mathsf{C}$ factors through $\iota$ uniquely.
    Conversely, the \emph{limit} of $F$ is a functor $\underline{\lim{F}}$ for some $\lim{F}\in\Obj\mathsf{C}$ together with a natural transformation $\pi:\underline{\lim{F}}\to F$, charachterized by following universal property:
    For any natural transformation $\underline{A}$ for some $A\in\Obj\mathsf{C}$ factors through $\pi$ uniquely.
\end{defn}

\begin{rem}
    One can write the definition of colimits and limits explicitly.
    For $\colim{F}$ under above settings, this means the following data:
    Let $i\in\mathsf{I}$, we map $i$ to $F(i)$, $f:i\to j$ to $F(f):F(i)\to F(j)$, and have the \emph{structure morphism} $\iota_i:F(i)\to\colim{F}$.
    Moreover, the universal property turns to be that for any given $A\in\Obj\mathsf{C}$ and morphisms $\varphi_i:F(i)\to A$ for each $i\in\Obj\mathsf{C}$, there exists a unique morphism $\tilde\varphi:\colim{F}\to A$, such that for any $f:i\to j$, the following diagram commutes:
    \[\begin{tikzcd}
        F(i) \ar[rr, "{F(f)}"] \ar[rd, "{\iota_i}"] \ar[rdd, "{\varphi_i}"'] & \ & F(j) \ar[ld, "{\iota_j}"'] \ar[ldd, "{\varphi_j}"] \\
        \ & \colim{F} \ar[d, dashed, "{\tilde\varphi}"] & \ \\
        \ & A. & \
    \end{tikzcd}\]
    The charachterization for limits is nothing but reversing arrows.
\end{rem}

\begin{nota}
    If $\mathsf{I}$ turns to be an index set, we often write limit and colimit of $F$ to be
    \[\lim_{i\in\mathsf{I}}F(i),\quad\colim_{i\in\mathsf{I}}F(i).\]
    Or we write
    \[\lim_{U\in\mathsf{C}}U,\quad\colim_{U\in\mathsf{C}}U,\]
    if $\mathsf{C}$ happens to be the ``image'' of $F$.
\end{nota}

\begin{defn}
    If $\mathsf{I}$ is a small discrete category, that is, we have morphisms
    \[\Hom_\mathsf{I}(i,j)=\begin{cases}
        \{\id\}, & i=j,\\
        \varnothing, & i\neq j,
    \end{cases}\]
    then for $F\in\Fun(\mathsf{I},\mathsf{C})$, the limit of $F$ is called \emph{product} of $F$, and the colimit is called \emph{coproduct} of $F$.
\end{defn}

We have the following two well known propositions, whose proofs are direct.
\begin{prop}
    The product exists in categories $\mathsf{Set}$, $\mathsf{Grp}$, $\mathsf{CRing}$, $\mathsf{Mod}_A$ and $\mathsf{Top}$, whose products are Cartesian products with projections.
    Notice that for $\mathsf{Grp}$, $\mathsf{CRing}$ and $\mathsf{Mod}_A$, projections are indeed homomorphisms.
    For $\mathsf{Top}$, let $\{X_i\}_{i\in\mathsf{I}}$ be topological spaces with corresponding basis $B_i$ for each $i\in\mathsf{I}$, then the topology of $\prod_{i\in\mathsf{I}}X_i$ is generated by
    \[\left\{\left.\prod_{i\in\mathsf{I}}U_i\subset\prod_{i\in\mathsf{I}}X_i\right|\ U_i\neq X_i\text{ for finite many }i\in\mathsf{I}\right\}.\]
\end{prop}

\begin{prop}
    The coproduct exists in categories
    \begin{enumerate}[(1)]
        \item $\mathsf{Set}$ and $\mathsf{Top}$, whose coproducts are disjoint unions with embeddings.
        In $\mathsf{Top}$, for topological spaces $\{X_i\}_{i\in\mathsf{I}}$, a subset $U\subset\coprod_{i\in\mathsf{I}}X_i$ is open if and only if $U\cap X_i$ is open for $i\in\mathsf{I}$.
        \item $\mathsf{Grp}$, whose coproducts are free products with embeddings.
        \item $\mathsf{Mod}_A$, whose coproducts are direct sums with embeddings.
        In particular, $\mathsf{Ab}=\mathsf{Mod}_\mathbb{Z}$, coproducts exist in the category of abelian groups.
    \end{enumerate}
\end{prop}

\begin{rem}\label{need ref 1}
    We will discuss the coproducts in $\mathsf{CRing}$ in commutative algebra chapter. % give a ref after finishing tensor product
\end{rem}